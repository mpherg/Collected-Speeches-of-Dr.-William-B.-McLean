\documentclass[twocolumn]{memoir}
\usepackage[utf8]{inputenc}
\usepackage[english]{babel}
\usepackage{csquotes}
\usepackage[style=numeric]{biblatex}
\bibliography{references.bib}

\title{%
  IS RESEARCH DIFFERENT \\
  \small Invited talk by Wm. B. McLean, Technical Director, U. S. Naval Ordnance Test Station, China Lake, California, to the Bendix Management Club, Los Angeles, California, 26 April 1952
}

\author{Wm. B. McLean}
\date{26 April 1952}

\begin{document}

\maketitle

I have long been of the belief that research people react differently than other categories of people and that therefore the management of research organizations must be handled differently from other types of organizations. I recently came across an article by Rensis Likert, in the March issue of ``International Science and Technology'' \cite{likert}, which attracted my interest by agreeing with my views. Dr. Likert begins by listing the observations and results of an extensive program by the Institute for Social Research of the University of Michigan. These seem to be the kinds of things which a scientist takes for granted as being necessary for his effective performance. They include:

\begin{enumerate}
  \item Good communications and mutual respect between members of the organization.
  \item Frequent communication with colleagues from other fields and with those who have different methods of approach.
  \item A high degree of self-confidence among the scientists so that they can maintain their independence of mind, even in the face of different opinions by their colleagues.
  \item Scientists and engineers, who see their administrative chief often, perform rather better than those who do not. 
  \item A scientist seems to need a high degree of self-determination combined with free access to someone in authority. 
\end{enumerate}

It was encouraging to me to see that the results of social research agreed so closely with the criteria which I would set from my own experience as being the conditions which would lead me to the greatest productivity. Dr. Likert reaches a general conclusion which I would like to quote:

\begin{displayquote}
Scientists and engineers are likely to be most creative when their supervision is such that they feel substantial freedom in their work---in selecting their problems and goals, in deciding on the approach to achievement, and in interpreting their data---and when they have frequent interaction with their superior. These findings are valid for the administration of basic research, developmental research, and engineering.
\end{displayquote}

I believe that a conclusion of this type will find little objection from most working scientists and engineers. When these same scientists and engineers become managers, however, their problems in establishing these conditions for others become extreme. They are trained to believe that a classical organization is supposed to have a definite mission, with its progress planned and scheduled. In managing an organization, we should be able to establish a budget with definite milestones to check our progress against our rate of expenditures of funds. Can we work in a large organization and accomplish its definite goals if we leave each individual scientist and engineer freedom to select his own individual problems and decide on his own approach to solutions?

As the head of a relatively large research and development organization, I feel these dual pressures strongly. I find that I have many ideas which I would like to have tried as a working scientist. I have very definite ideas about details of design for such ideas. I also have methods of approach to the solution of the problems which may be found to exist. I will have to confess that at the present time I have not found a satisfactory technique by which I can get the scientists and engineers in my organization to carry out my programs in the way that I would do them. Does this mean that the organization is a failure, or does it mean that I have failed as a manager to make my requests sufficiently explicit and direct? I do not believe that either alternative is true. I am simply caught in the position of trying to perform simultaneously two functions which are incompatible. As a scientist, when I have an idea and outline a specific design and program for carrying it through, I am performing a creative function which requires the freedom of choice, the responsibility for errors, and the necessity to carry the job through to completion, including the checking of data and correction of errors as they become apparent in the methods of approach. If I want to work as a project engineer, the organization will help me accomplish the goals which I have chosen. But my position as manager of the organization commands only slightly more assistance than any other project engineer and I have to continue to assume the responsibility for success or failure of the specific designs. Also, as a senior manager in the organization, there are some additional hazards to carrying a job through successfully because of the competitive position with other project engineers.

If, on the other hand, as a manager I want someone else to assume the responsibility for carrying through a specific problem, then I must get help from the organization at the early planning stage. I must find a man who will assume responsibility for each part. He must feel that he has the freedom to choose his program and that he is carrying it out along the lines which he believes are best. Any restriction of this freedom automatically frees him from the responsibility of making the final gadget work. In a large organization or project it is difficult to achieve the type of participative management required so that each man working on a part of the system feels complete personal responsibility for his piece of the whole program. He must have enough information and enough understanding of the goals of the total project to make his piece fit in the optimum manner. It is in accomplishing this general under standing that communications and mutual respect show their important influence on the total effectiveness of the organization.

The difficulty of maintaining effective communication with increasing size of an organization is the reason, I believe, that organizations tend to become less efficient as they become larger. Dr. R. B. Kershner of the Applied Physics Laboratory, Johns Hopkins University has written a very interesting paper on the optimum size of organization for any given job \cite{kershner}.  He plots the time to accomplish a given objective against the number of people assigned to the task and shows that the curve has a minimum value.  With too few people assigned, the job moves too slowly to maintain the interests of the people and their sense of accomplishment. As a result, a long time is required to finish the job. If the number of people is increased beyond the optimum, competition for the jobs available becomes keen. Communications begin to fall off. The understanding of what is to be accomplished becomes more remote. The need for specific, definite specifications becomes greater. And finally, the ability of each engineer to participate in setting up the goals toward which he is working, and his contribution to the total design, becomes less with a resulting loss of interest. Tension within such an overstaffed organization grows, mistakes become more common, and the ability to try new things which might lead to significant short cuts becomes entirely too risky. The need for more coordination and more planning as the program lags becomes more apparent. The system is self-accelerating in that, as more coordinators are added, the engineers and scientists have less opportunity to provide feedback into the setting of specifications; thus, progress toward the final goal is further delayed. If we want to avoid these difficulties and have a participative type of operation, we should, as managers, try to do every job with an organization which is at the optimum, and this usually means the smallest size for its effective completion. If we increase beyond the optimum size, however, the forces become such as to automatically justify further increases in size.

How then can we tell whether we are on the high or the low side of the optimum organization? I have proposed that one technique which might be interesting to try would be to start each important program at levels separated by an order of magnitude. If we can justify the funds on the basis of the high level program, then the expenditures for the low level program will seem to be insignificant by comparison. I suspect that most of our military programs are now on the high side of the optimum expressed by Kershner's curve. If this is true, we should expect in many cases that a competitive program, working toward a given objective, which is funded at 10\% of the present going program should have a reasonable probability of coming through in a shorter time and with a better final product. The product will be better because of the greater integration of the design which is possible in the smaller organization. The homogeneity of the small group, and the spirit of competition with the larger group will engender a group motivation which should lead to a high degree of creativity and cooperation.

All of these ideas for improving research or development organization, or for improving the working conditions of scientists and engineers, tend to become very controversial. We are continually faced with the question of determining whether any change we institute improves or degrades the organization, or the effectiveness of the individual. Dr. Likert suggests that as we become able to measure the improvements in an organization, by the periodic measurement of such things as employee attitudes, motivations, and the adequacy and accuracy of internal communications, we will gradually collect the data which will make it obvious that we should shift from what he calls the traditional management system over to the participative management technique. I believe that all of the specific measurement which he mentions, such as employee attitudes and motivations, or the accuracy of communications, can be grouped under one measurement which is relatively easy to make---the only defect in its general acceptance seems to be that it has an implication of frivolity. It is my conviction that all of the elements of an effective, creative, and productive organization can be measured under the single heading of a determination of whether the people making up this organization are enjoying themselves and are having fun doing their work. On the contrary, a poor organization is one in which fear is the guiding motivation and unhappiness with working conditions is apparent everywhere. It is, of course, possible to imagine situations which involve considerable enjoyment but very little productivity. I do not believe, however, that this type of operation can be maintained on a long term basis. I think it is the basic nature of man to enjoy being productive. His basic reasons for organizing are to increase his individual productivity. When the organization grows to the point where purely organizational goals become the dominant motivation, and man must serve the organization rather than have the organization provide the tools for man's creative expression of his desire to produce, then I believe we run the risk that the motivation of joy in achievement will be replaced by the motivation of fear of failure.

Dr. Likert concludes his paper by saying that their research suggests that all types of people will benefit from the same management required for engineers and scientists. Better management will concentrate on individual creativity of all kinds. This statement led me to my title---``Is Research Different.'' Is a participative management something which should apply only to scientists and engineers? As a technical man, I am sure that my productivity will vanish if I do not have such management, but as a compromise with the traditional theories of management, I am happy to sacrifice the rest of the organization to one way planning, rather than participation, if this will leave the technical work free. However, even in the Preamble to the Constitution we have a general statement that the largest organization, the Government, should be designed so as to provide the tools by which each man can achieve the expression of his goals to the limit of his ability. The opposite governmental system, in which the goals of the organization predominate and the man must subdue his individual interests and desires for the good of the total organization, is represented by the Communistic-type of society. It will lead eventually to a communal type of life which is well exemplified in its advanced forms by the bees and the ants. This type of organization can be very stable but it cannot achieve real adaptive progress because individual creativity has vanished for the good of the smoothly running organization.

I hope that we as a Nation can choose in the management of our businesses and our military programs the type of management which maximizes enjoyment, participation, and the contributions of individual creativity, rather than the type of management whose goals and objectives are set from the top and which is budgeted, planned, and integrated to achieve objectives on schedule without consideration of possible creative inputs. One type of management will strengthen what we have variously called ``The Free Competitive System,'' ``The American Way of Life,'' or ``Life, Liberty, and the Pursuit of Happiness.'' The other type of management by overinsistence on the importance of budget and schedule, comes perilously close to conditioning us to the type of organization which believes that man's highest goal is to achieve and surpass through successive five and ten year plans.

\printbibliography

\end{document}
