\documentclass{memoir}
\usepackage[utf8]{inputenc}
\usepackage[english]{babel}
\usepackage{csquotes}
\usepackage[style=numeric]{biblatex}
\bibliography{references.bib}

\title{%
  THE SIDEWINDER MISSILE PROGRAM \\
  \large by \\
         WM. B. McLEAN Technical Director \\
         U.S. NAVAL ORDNANCE TEST STATION \\
         China Lake, California \\
         \vspace{12cm}
  \small Presented at the National Advanced-Technology Management Conference held in Seattle, Washington, 4-7 September 1962
}

\date{5 September 1962}

\begin{document}

\maketitle

Presented at the National Advanced-Technology Management Conference, 5 September 1962, Seattle, Washington

\vspace{5mm}

\begin{tabular}{ll}
The SIDEWINDER Missile Program - & \parbox[t]{8cm}{
                                   Wm. B. McLean, Technical Director \\
                                   U. S. Naval Ordnance Test Station \\
                                   China Lake, California}
\end{tabular}

\vspace{5mm}

Ladies and Gentlemen:

When I was placed on the program immediately following Admiral Raborn and Dr. Burris speaking on POLARIS, I am sure it was more for the purpose of providing contrast, rather than to group similar programs. SIDEWINDER and POLARIS are certainly quite different---in size of the job, the amount of money involved, and in the kinds of management techniques employed. In the presence of this company, I have some of the feelings of the longhaired artist, who paints just for the fun of it, attending a conference to discuss the techniques of commercial advertising art. We were not commissioned to design SIDEWINDER; we had no externally imposed specifications; and we started with no timescales other than those imposed by competition. Our prime motivation was to avoid the construction of the Aircraft Fire Control System Mk8 whose purpose was to fire unguided air-to-air rockets. This system we felt would produce an inferior result compared to a properly designed homing missile. We felt we had an inspiration with regard to a method of approach to such a missile design which would be fun and a challenge. We also believed there was a good chance that this missile could be made to work as simply and more effectively than the fire control system for unguided rockets.

Although designing missiles was new for our group, we did have some pertinent background experience. I had been associated with the problems of packaging electronic equipment in a small space, and the requirements for high reliability and producibility, through working on the VT fuze programs during World War II. I also had an early introduction to the problems of missile control by acting as a consultant on the design of some gyros and actuation systems for the BAT missile when it was being developed by MIT and the Bureau of Standards. From 1943 until 1948, I was working with a group trying to improve the sighting systems for air-launched rockets. Our first problem was to localize the major source of error: pilot, computer, aircraft, or target motion. In the course of this study we developed ranges and range techniques and acquired a deeper appreciation of what pilots, aircraft, and aircraft crews could be expected to do as well as the things which they probably would not be able to do. Since our fire control system was also to provide for the firing of missiles, we were able to visit many of the activities engaged in the design of missiles at that time. This included the German V2 scientists at White Sands; the Hermes people at GE; the SPARROW organization at Sperry; the METEOR work at MIT; the BUMBLEBEE group at the Applied Physics Laboratory, Johns Hopkins University; the DOVE work at Eastman Kodak; and the FALCON work at Hughes Aircraft. We owe all these programs a debt of gratitude because at each place we discussed the relative merits of unguided rockets versus the problems of missiles and collected data on the difficulties which each organization was encountering in the design of their specific missiles program.

By 1948, we had achieved good measurements of all of the contributing errors for unguided rockets launched from aircraft. Our results showed that such rockets can be made very effective against non-maneuvering targets. Nevertheless, if the target makes an unpredictable maneuver after the rocket has been fired, the flight time of the rocket is usually so great that the unpredictable target motion after firing can produce an error more than three times greater than the sum of all other contributing errors. We also learned a great deal about the difficulties of keeping electronic equipment operating in an aircraft. The only good answer to the air-to-air problem seemed to us to be a guided missile which could solve the fire control problems as it progressed. Also, if we could keep most of the fire control circuitry in the missile, the maintenance of the fire control system would be solved, at least in wartime, by shooting it.

Unfortunately, we reached this decision at a time (1948) when the anti-missile sentiment in the country was very high. There was great disappointment in the fact that the missile programs were progressing so slowly and that they were so expensive. Great effort was being expended to cut back on missile developments. Every time we mentioned the desirability of shifting from unguided rockets to a guided missile, we ran into some variant of the following list of missile deficiencies:

\begin{enumerate}
  \item Missiles are prohibitively expensive. It will never be possible to procure them in sufficient quantities to use them in combat.
  \item Missiles will be impossible to maintain in the field due to their complexity and tremendous requirements for trained personnel.
  \item The prefiring preparations, such as warm-up time and gain setting required for missiles, are not compatible with the targets of surprise and opportunity which are normally encountered in air-to-air and air-to-ground combat.
  \item The fire control systems required for the launching of missiles are complex, or more complex, than those required for unguided rockets. No problems are solved by adding a fire control computer in the missile itself.
  \item Guided Missiles are too large and cannot be used on existing aircraft. The requirement for special missile aircraft will always result in most of the aircraft firing unguided rockets.
\end{enumerate}

This series of objections, expressed many times and in a multitude of variations for the purpose of keeping us out of the missile development operation, constituted our design objectives for the SIDEWINDER system. Many hours of thought and discussion were required before we finally felt we had a design with some chance of being acceptable to the people who were not in favor of guided missiles. Therefore, one of the major differences in the SIDEWINDER program, compared to other missile programs, was that it was designed to please people who in general were \underline{against} such systems, rather than the people who were in \underline{favor} of them.

I suspect that we might find here a general management principle which is that a statement of the problems or the objections to an existing system will produce a more creative approach to the design of a new system than will be achieved by a set of definitive specifications for the new system. Valid objections leave many approaches open. Specifications tend to channelize thinking along the lines of a single approach.

The design we originally proposed for SIDEWINDER, in order to meet the existing objections to missiles, had many elements which had not been tested at that time, such as torque balance control on the canard wings, rollerons for roll stability, a propellant-driven power supply, a hot gas servo, and a gyro tracking system which was independent of roll rate. If the SIDEWINDER missile had been a crash program, any one of these elements would have been too risky to include in a system which had the need to proceed on schedule. Since the management pressures on SIDEWINDER were, at most, permissive, we had the opportunity to carry out feasibility tests at the various critical components before proceeding with a complete missile assembly.

We might have fallen into the trap which is quite common for low pressure programs of never finishing the feasibility studies by always seeing the opportunity for improvement. If one becomes so trapped, a final design will never result. The SIDEWINDER program very fortunately, in addition to having a permissive atmosphere from our immediate supervisors, also had a strong active opposition from some of the higher levels of management. About every three months we had a committee of experts to investigate one of the critical areas in the design to see if this area was not sufficiently shaky to merit cancellation of the whole program. This provided a powerful incentive to our group to complete the feasibility investigations and come up with a firm, proven and field tested design in each critical area prior to the arrival of each group of specialists.

By 1953 we had our first successful shot, and by 1956 the missile was in production and was seeing service use by the fleet. The missile was successful in that it avoided most of the original objections to its use in combat situations which were given by the antimissile people. Now SIDEWINDER has been accepted for service use and has even been used under rather unusual combat conditions by foreign services with a minimum of training time.

\underline{MANAGEMENT TECHNIQUES}

Now that I nave said a few words about the initial conception of SIDEWINDER, and something about its history, I will proceed to the things which are more pertinent to this conference; that is, the techniques by which the program was managed. The management of SIDEWINDER was relatively easy because of the organizational setup. We had a rather small number of good people who were highly dedicated to getting the job done and who worked closely together so that they had a good appreciation of the over-all problems. They had immediately available all of the tools needed to do a complete job, from basic research through testing plus continued contacts with fleet personnel as to which techniques were most likely to be acceptable by the people who would be using the equipment.

Communications were facilitated by the fact that the working group was isolated in a small community in the desert about 150 miles from the nearest large city. People could and did communicate with each other all day, through the cocktail hour, and for as long as the parties lasted at night. This isolation in a location where the job could be performed provided large measures of the intimate communication which is so essential for getting any major job completed. In all honesty, however, I will have to report that this particular technique for communication generates some family strains and pressures. The wives tend to be less enthusiastic about continuous attention to the work than the men.

In looking back over the program, the single most important abstraction I would draw is with regard to the importance of not starting too fast. At the start of the SIDEWINDER program, I personally spent nearly three years on a part-time basis in the process of considering possibilities---mentally arranging them into a missile, checking the tradeoffs, and trying to think of other methods of arrangement which would make the final design more acceptable to the user. At this stage of the development, reorientation of the program is easy. A complete reorganization of the internal workings of the missile can be accomplished literally in the time that is required to think of it. I believe that this process, by which one man gets fairly clearly in mind a picture of what he would like to produce and the reasons for selecting one set from a multitude of possible choices, is a very important step in the accomplishment of a satisfactory final product. This man must perform very much the same functions as an architect in the construction of a building. In our present method of budgeting funds, the function of this man is quite frequently lost. If he ever did exist at the planning stage, he will almost certainly be lost before the project reaches the stage where the completed design must be put down on paper.

As the SIDEWINDER design was committed to paper, we began to lose flexibility, and the critical elements of the design became apparent. The need for both the construction of test hardware and the carrying out of critical tests grew, and with this growth the organization increased in size. Timescales became interdependent, and at least informal schedules for the various parts of the program had to be made in order to allow people to work on their particular parts of the program independently. We were very fortunate in not having to build the complete design as an entity and send it away to another organization for test. We were able to construct crucial parts of the system and test them directly before going on to the more complicated parts. We were fortunate in having a relatively small group of engineers working together on pieces of hardware that were carried through all of the stages from design to final test. This provided a very important experience in training them not to overlook necessary elements when the tests became more complicated and the organization larger.

During the slow starting period the small group was able to establish their own goals. Good technical people like to believe that they are doing their own planning. This becomes increasingly difficult as the size of the organization increases. They begin to see that they are caught in someone else's plans and, as a result, they may lose the high degree of self-confidence and initiative which are perhaps at the root of the successes which have distinguished them as outstanding people.

Before the first successful shot was completed, we had between 250 and 300 people working on the project at the Naval Ordnance Test Station. We were monitoring work of at least four other government installations, as well as the prime guidance contractor and about ten or twelve other industrial organizations. We found it necessary to hold quarterly and then monthly meetings with representatives from all of the groups working on the project. These meetings quite often generated lots of heat and very little light. The resistance to change increased remarkably. The differences of opinion would never have been resolved if it had not been for the experience of the original, small integrated group that was familiar with the process of taking things out for test. Theoretical arguments and calculations will resolve many problems; however, Mother Nature, if asked questions in the right form, performs an arbitration function on the test ranges which is beautiful in its conclusiveness.

I am sure you are all familiar with the difficulties of making changes when l,000 or more detailed drawings are involved, and with the inertia of a production line set up with programmed belts and assembly fixtures for all of its people. At this point, the design is finished, and creativity in design had better be saved for the next project. The maximum use of creativity must be confined to making the machine work with the fewest possible changes in any of the parts. We were fortunate in having a design that could be produced with only minor changes. If this were found not to be true and if the basic design were found to be unsound, then time would probably be saved by scrapping the whole program and starting over. However, the investment up to this point is usually too great to permit this solution. We usually compromise on a less than optimum design and shift our responsibilities to fleet maintenance and operation.

Let me review what I believe to be the important steps in the management of a successful design project for a new system. \underline{First}, I believe it is essential to have a man who can visualize in some detail what he would like to create and who has carefully thought through the problems associated with the creation of his system. He will perform, with respect to his design, much the same functions which an architect performs in the design of a complete structure. \underline{Second}, the designer, or architect, needs to interpret his vision of the complete system by sketches and rough drawings which can be used by other engineers and technical people to do more detailed design and construction of the parts so that they will fit into an integrated whole. Each of the additional people must understand the complete design and must communicate frequently with the architect to be sure that the parts fit properly into the whole system. \underline{Third}, the organization can be expanded and the production of subsystems for test and evaluation can proceed. While this is in progress, the man in charge should make frequent visits to check on all of the components to see if they are progressing in the way in which he had visualized them and that they will not distort the final product. He also must perform the function of rejuggling the compromises as some parts of the system be come easier, and other parts become more difficult.

\underline{Finally}, the components are assembled into a working and tested model of the complete system. Here we reach the first point at which the services of the master designer, or architect, might be dispensed with and his design turned over to others for production. Such a shift becomes possible at this point, but is usually not desirable. Most designers would like to continue contact throughout production, although their interest will decline as the problems disappear.

\underline{MILITARY DESIGN AS A SPECIAL CASE}

Are there elements in our present \underline{government} management system which tend to prevent this logical prosecution of a project by picking a man, letting him think through a problem, test the areas he finds critical, construct and test a prototype, and finally, put the prototype into production? I think there are several.

The first roadblock in this process appears to come from our inability to pick a designer. This choice has some of the same pitfalls inherent in choosing a man to do any other form of artistic endeavor, such as architecture, painting, or composing music. The creative man in any field tends to be more interested in the process of creation than he is in the monetary rewards which may result from this process. In fact, it is only through the \underline{use}, not the creation, of a creative product that any large monetary return can be realized. Artists tend to be more concerned about creation of something new than they are about its eventual use. Artists, such as painters, musicians, and sculptors can usually express their ideas with very little investment in tools or capital equipment. The opportunity for self expression is therefore available in these fields to any number of people who have the urge to try them without the concurrent need for immediate use and profit. Large numbers of people can try these forms of art, and we learn to judge the best and discard the rest. Creative technical design, however, involves a considerably higher expenditure of funds and we therefore feel much more compulsion to carry out this process in such a way that it will always be successful and show a net profit. The number of creative designers who can compete on a directly comparative basis is therefore extremely limited. Therefore, the number of art critics who have been developed that are capable of judging creative designers is almost non-existent. We need to find a technique by which more unrecognized creative designers can test their skill and develop their talents without the danger of complete and disastrous failure, both to themselves and their associates. We need also the skill to recognize that a design may be poor even though it works.

The second government management problem comes in providing protection to our selected designer so that he can think through his problem and exercise his creative talent prior to being committed to a course of action. The mechanism in government for the establishment of a budget, and shepherding it through uncounted discussions on relative priorities, tends to draw a go-no-go line with respect to every project. A project is worthless and not meriting of any support until it is possible to build up enough arguments for its merit, and enough information on its need, that it crosses the threshold of recognition and its accomplishment suddenly becomes a national emergency. In the crash activity which results, the men who have been spending the time thinking the problem through have either be come discouraged and started a new project, or they are forgotten in the confusion which accompanies the initiation of a crash program. The time to think is at a minimum in crash programs.

A third management problem which prevents the orderly progress of a development is the degree of commitment to a design which is inherent in our governmental approval process. When any project breaks over the threshold required to get sufficient attention in the budgetary process, a rush request goes out, formally and informally, to 50 or 100 organizations to complete a design competition with a deadline of perhaps two weeks. The design proposal must carry with it not only the detailed method of approach, but also a time schedule and a budget. Any of us who have been through the process know that this degree of commitment to design, time, and budget makes it very embarrassing to change the design. Change is difficult even when test of the critical elements shows that the original concept was unworkable. It is even more difficult to make changes which are purely for the purpose of improving the esthetic appeal. In fact, I imagine that even mentioning the possibility that a design should have esthetic appeal will cause some people to think that I am completely impractical.

The final government management problem lies in the need to skip an orderly development process when carrying out a crash program. We finally convince ourselves that a program is sufficiently urgent to get started and we now must, of necessity, complete it in the minimum conceivable time. This means that there will obviously be no time for the construction and test of a prototype. We must run the calculated risk of initiating the procurement of long lead time items, and the final design will thus be biased in order to use them. We must start the construction of our test ranges and our production facilities while our design is still struggling to be born. Since specifications for ranges and production facilities are easy to write, their construction will proceed rapidly, and their completion may determine what can be tested or produced.

It is easy to state what is required for good management of a program, and it is equally easy to see that these elements will be difficult to achieve in a real political atmosphere. What we need is the invention of a management process which will satisfy the competing technical and political requirements. I would like to propose such an invention for your consideration.

Dr. Kershner of the Applied Physics Laboratory, Johns Hopkins University has written a very provocative paper \cite{kershner}  which has laid the ground work for this proposed management invention. Dr. Kershner points out that for every project there is an optimum size organization. He postulates that if we plot the time necessary to accomplish a job, versus the number of people employed on the job, we will find that this curve will at some point have a minimum value in the time required to accomplish the job. Determination of the number of people representing the minimum value for different kinds of projects is the manager's dilemma. His job is particularly complicated by the fact that, at every point on the curve, all the people involved in any project believe that the organization is too small. In fact, as the working force passes the minimum point on the curve, and the rate of progress begins to drop, the perceived need for more people and more liaison increases rapidly. And, as Dr. Kershner points out, more engineers on the project can invent more avenues of approach, and more techniques to try. The maintenance of coordination between all of these different possibilities becomes a function which again requires more people and more paper work.

I would propose that the determination of the minimum size organization for any particular job is a problem which can only be accomplished by an experimental approach. When we are working in well-travelled areas, such as producing new models of automobiles or constructing houses, we have established norms which serve as guides on future jobs. Or if we have a competitive operation, then the fact of competition will in time bring the size of the organization to its optimum value. In the development of new military equipment, we need a substitute for standard norms or for the operation of a competitive process.

I believe that such a management technique can be accomplished if, for every military program started in an area of work where standards are not available, we set up two competing approaches which are separated in funding or manpower by an order of magnitude. I am not prepared to argue that this is the best separation. Any other separation would also provide the kind of data which we are seeking. However, I believe that on most of our military programs our departure from the optimum size organization may be in error by at least an order of magnitude. Any smaller separation of the two programs is therefore not likely to produce results with the maximum observable difference which will make it easy to determine the slope of our curve and, therefore, our position on it.

In operation, whenever we are forced by political pressures to start a crash program, we should select a prime contractor and start funding him at whatever level seems politically expedient. We should then survey the remaining people and find those who believe that they have a technical idea which is sufficiently novel to allow them to have a reasonable expectation of competing with the prime contractor at a cost which is one-tenth the one at which he is actually funded. An important element of this process is the necessity of maintaining the ratio between the two projects as they progress. Neither project really has a good appreciation of the technical difficulties which will be encountered if the work is in a really new area. Both will feel that additional funding is essential. However, our management techniques should be quite rigid in maintaining a fixed ratio between the two programs no matter what arguments for additional funding may arise. I think you can probably see that one of the effects of maintaining this ratio inviolate will be to put most of the burden for justifying additional funding on the program which is proceeding with too much manpower. This will improve total net progress by providing jobs for the excess people.

If both programs actually lie beyond the optimum number of people for the most efficient accomplishment of the job, then the ten percent program will be making the higher rate of progress. We will have automatically built into the system the management tools required to keep the crash program from expanding in a disastrous manner.

If we are able to attain the very desirable condition in which the ten percent program is just slightly smaller than the optimum manpower or funding level, than I suspect that the condition can arise where both programs will be observed to be making approximately equal progress. At this point, I suspect that it will take all of our persuasiveness as managers to convince the budget analysts that it is essential to continue the program funded at the high level. Its function as a tool to shield the low-funded program from the political pressures and military requirements will be hard to explain. However, without this protection, the low-funded program will be unable to take the high risk ventures which are essential if our accomplishments are to be great compared to our effort. We need both types of programs in order to be able to achieve real technical progress in a framework of a real political environment.

In essence, my proposal for the management of military programs is to attack each objective with both a large program, which will provide a safe, scheduled and well funded route toward the objectives, and a more risky venture funded at ten percent of the cost where we can try out the talents of our creative designers without forcing them to risk the political safety of the nation while they are taking chances.

I believe that SIDEWINDER was such a ten percent program, and that it owes a large measure of its success to the shield provided it from political pressures by its competitors, FALCON, METEOR, and SPARROW.

I believe that SIDEWINDER was produced in a nearly ideal management setting and that the only defects I can find in this management system would be rectified if this type of program were to be (a) recognized as a management device, and (b) funded at the ten percent level without the need for wasting time on justifications for more money.

If such a management tool is ever accepted, l personally hope that I can always find an opportunity to work on the ten percent end of the programs where the creative technical approaches sometimes will have the high payoff, rather than on the one hundred percent end where the prime emphasis must be on setting and meeting schedules imposed because we live in a real world with real political needs.

Both types of program are essential and the people exist who will believe in and enjoy one or the other approach. The competition between the programs should provide valuable incentives for both groups, always resulting in reduced total costs and sometimes with exceptional products when a risky and creative venture succeeds.

Thank you.

\printbibliography

\end{document}
