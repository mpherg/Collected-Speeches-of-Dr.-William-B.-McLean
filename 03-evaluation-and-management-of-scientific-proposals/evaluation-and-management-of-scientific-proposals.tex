\documentclass{memoir}
\usepackage[utf8]{inputenc}
\usepackage[english]{babel}
\usepackage{csquotes}

\title{%
  DIGEST OF PROCEEDINGS \\
  of the \\
  INSTITUTE \\
  FOR \\
  CAREER SCIENCE EXECUTIVES \\
  \large October 12 --- 20, 1960 \\
         Stone House \\
         National Institutes of Health \\
         Bethesda, Maryland
}

\begin{document}
\maketitle

\begin{vplace}[0.7]
\begin{center}
FOREWARD
\end{center}
The Institute for Career Science Executives was designed to promote the effectiveness of Federal scientist-executives through the study of important concepts and current issues relating to the organization and administration of scientific and related activities of the Federal Government.

These summaries were prepared by the participants and represent their views of the most significant points made by the speakers.
\end{vplace}

\pagebreak

\begin{center}
Summary of Panel Discussion \\
On \\
EVALUATION AND MANAGEMENT OF SCIENTIFIC PROPOSALS
\end{center}

\vspace{1cm}

\begin{tabular}{rp{8cm}}
Members of Panel: & \parbox[t]{8cm}{
                    Mr. James F. Kelly                                           \\
                    Director                                                     \\
                    Office of Financial Management                               \\
                    Department of Health, Education, and Welfare                 \\
                    \\
                    M. Willis R. Shapley                                         \\
                    Chief, Air Force Section                                     \\
                    Military Division                                            \\
                    Bureau of the Budget                                         \\
                    \\
                    Dr. William B. McLean                                        \\
                    U. S. Naval Ordnance                                         \\
                    China Lake, California                                       \\
                    \\
                    Mr. DeMarquis Wyatt                                          \\
                    Technical Assistant to Director of Space Flight Development  \\
                    National Aeronautics and Space Administration}
\end{tabular}

\vspace{5mm}

\begin{tabular}{rp{8cm}}
Recording Team: & Dr. A. F. Bartsch, Dr. Jacob E. Dinger, Mr. Richard G. Grassy, and Mr. William Luzerne Lovejoy
\end{tabular}

\vspace{1cm}

The question ``How does a program administrator keep his research people working on questions to which we will need to have program answers?'', sometimes seems a problem because the scientist is motivated to slant the proposal for a research program along lines which may best ``sell'' the idea. This may be different from his true interests and the exact approach he ends up pursuing. The more persons who pass on a proposal, the more unlikely a novel idea will be accepted. In setting up a basic research program, emphasis should be on selecting good research people and not so much on proposals. A formula for generating good basic research is: ``Pick a good man, eye him for life, then support him with funds on that research in which he is willing to invest some of his own financial resources.'' This formula has the ingredient of supporting research in which the man is sufficiently enthusiastic to venture some personal risk. One example of such risk is his putting his reputation on the line when backing certain research.

The lines of effort to be followed by an organization is often considered an administrative matter---outside the initiative and freedom of the scientist. Every activity obviously has a mission. Short of establishment of a Federal ``Institute of Research'', Government organizations are established to carry out missions not directed primarily at conduct of general research. The scientists' choice of research area is therefore academic, and imposition of controls to insure compatibility of research with agency mission is the function of administration. It is necessary to consider the people who will do the actual work. Evaluation of their proposals for research will include, as a primary factor, consideration of their qualifications---principally of their prior work. Such evaluations are best made at lower management levels, close to the men and the work.

Many of the administrative problems of keeping the program in line with the agency mission can be solved or avoided by inculcation of a team spirit oriented toward solution of agency problems. Development of such a spirit, and the leading of individualists along lines beneficial to the agency, are the responsibility of management.

Basically the issue of keeping the researcher working toward needed program answers is a problem of how to work with people. This problem occurs in all fields, but research does have some unique problems of its own. A notable one is failure of communications between the researcher and management. The objective in hiring the man, which must be related to the mission, is not always clearly emphasized. Tangential interests of the scientist sometimes lure him away from the mission.

Insofar as basic research is concerned, once a program has been chosen by management, little detailed direction should be provided. The individual scientist should be given free rein to attack the matter in the way he considers best. Even in basic research, however, complete freedom may not be entirely justified. The very selection of the program to be undertaken removes, to some extent, the ``free rein'' atmosphere. There must be a motivation toward some goal, and in many cases there must be a reason given which permits the scientist to rationalize directing his talents to the program. For example, scientists working on the Manhattan Project often asked themselves later, ``Was my participation in this project morally justified?'' To answer such questions, national goals or needs, such as national security, must be crystallized in meaningful terms. Here management must take a part and must draw upon the talents and thinking of those normally considered as being outside the scientific community.

In applied research, the problem is quite different. Here the goals are much more narrow and specific; they are formulated at a lower level, and they must be closely adhered to if the project is to be a success. The limits of freedom of action must be much narrower, but still the scientist must be allowed some latitude in which to exercise his own judgment. Management must achieve a situation where effort is channeled without stifling initiative and independent thinking.

As an interesting sidelight, the policy of the Bell Telephone Lab, as imparted to new employees, is ``You are free to work on anything, but remember that we are in the telephone business.''

Basic research and development in the same institution is a desirable thing. People are all different, and a wide spectrum of activity permits a greater opportunity for a given individual to find himself.

When basic research is separated from applied research, as in all living creatures, each part tries to grow back the missing part. Basic policy of NASA is, however, that research and development should not be intermingled. A key factor is a tendency of the best scientists and engineers to migrate, voluntarily or involuntarily, into the development programs. This policy has not been entirely satisfactory because personal interests are not satisfied, and the cross-fertilization of research and development ideas is stifled. At the jet propulsion laboratory, a middle-course has been adopted; basic research of up to 25\% of manpower available is supported with technical program content controlled by the laboratory, subject to headquarters review.

The question of mixing basic and applied research in the same organization is one to which no hard and fast answer seems to be possible. It has been suggested as desirable that the individuals who generate a basic concept carry it through to the developmental stage, thus providing desirable continuity and insuring that the usefulness of the basic concept is fully exploited in a timely fashion. But this poses serious problems. Much time and effort can be expended in the ``tooling up'' and re-organization of an organization devoted to basic research to carry on applied and developmental research. There is also the problem of disposition of the developmental portion of the organization when the developmental work has been completed. Serious personnel adjustments may be necessary.

The universities apparently do not look favorably on mixing basic and applied research, and complain that the Government's insistence that they accept development contracts is seriously diluting or even eliminating their ability to do basic research. The universities seem to regard basic research as their proper sphere of operation.

Probably the most effective control of scientific research is maintained at the budget decision level. Here decisions are made as to where funds shall be spent and how much will be allotted. These decisions really mold, in a large degree, the course of future research. From that point on, decisions are largely confined to how the goals can best be achieved with the funds available. For example, the need to continue and expand basic research is well recognized generally. The questions to be resolved are where and how much. Top level management of a ``heads-up'' variety is required to recognize areas where research may provide a big pay-off and to provide flexibility that will permit the pursuit of this research. This flexibility must also apply to funding. One approach is represented by the DOD Emergency Fund which permits \$150,000,000 to be used to exploit unforeseen opportunities which may arise. Here, ``playing the hunch'' becomes important, and environment has a great bearing on this. For example, where a considerable investment has already been made, the question of playing ``hunches'' assumes an aspect different from that which existed when the original investment was made. However, the basic questions to be answered are essentially the same, i.e., how important is it, and how much are you willing to risk. Another aspect of environment is the nature of the climate within which an agency operates. Some are much more research oriented than others and give a more favorable outlook.

\end{document}
