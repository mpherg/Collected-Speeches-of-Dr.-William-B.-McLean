\documentclass[twocolumn]{memoir}
\usepackage[utf8]{inputenc}
\usepackage[english]{babel}
\usepackage{csquotes}

\title{%
  RESEARCH AND DEVELOPMENT OF MILITARY EQUIPMENT \\
  \large by \\
         WM. B. McLEAN, Technical Director \\
         U.S. Naval Ordnance Test Station \\
         China Lake, California \\
  \small (Invited lecture presented at the University of California, Berkeley, California, 17 April 1959, sponsored by the Engineering Lecture Committee, Department of Engineering.)
}

\author{Wm. B. McLean}
\date{17 April 1959}

\begin{document}

\maketitle

I have been asked to speak to you tonight about the problems of coordinating an engineering program in the research and development of military equipment.

In my opinion, a good engineering design is more of an art than it is a science in spite of the great emphasis which we have on technical calculations and on the improvement of engineering graphs and curves. It is an art primarily because the number of possibilities for a successful solution to a design problem becomes continually greater as our knowledge of equipments and techniques expands. The proper selection among these techniques is something which becomes more difficult to establish on a rigorous and rational basis the greater our knowledge of techniques and processes. I believe the failure to recognize the need for artistic choices in the design process is at the root of many of our management problems in our more complicated problems. We believe that because our scientific calculations are exact, the process of choice can also be exact.

The normal process used to accomplish a design, such as an intercontinental ballistic missile, is to authorize a contract and establish design specifications. A planned program with a fixed budget is needed before a man can even begin to think. From this point on, the imagination of the designer is limited and he feels it is his responsibility to meet these specifications even though minor changes would result in a much more effective overall design. It seems to me that the creation of a missile system would progress more effectively if it were recognized to have many of the same problems as the creation of a large mural painting. Many useful analogies might then result. The creation of a mural is obviously too large a job for one man and yet, at the same time, it must represent an integrated whole, rather than a collection of parts. In the case of the mural, we have adopted the practice of selecting a master artists whose responsibility is to conceive a picture in accord with the general message which is to be conveyed. He then uses his imagination, his understanding of the materials and tools available, and his knowledge of the abilities of his assistants to lay out an overall Design. Committees can review his work and make suggestions, but they cannot take over his responsibility for it. Once the general concept has been sketched out, many people can begin to work using their own specific abilities to fill in the various parts of the picture. As a result, we have an integrated creation that reflects primarily the skill, ability, and experience of the master artist, but which also uses the individual skills of his assistants to a maximum.

It appears to me that this same technique should be applied in the design of a missile system. We know generally what it should accomplish. We should select one man of demonstrated experience and ability in the field to conceive and layout the work necessary in the various component areas. These areas can then be filled in and, if proper coordination is maintained, the whole picture can be redirected and reoriented to achieve a better final result as limitations in some areas and advantages in other areas become more clear. I become exceedingly skeptical whenever I hear the phrase that "a missile is to be composed entirely of off-the-shelf items". I feel this design will probably have the same structural strength and beauty of conception that would be represented by a montage of photographs for the production of a mural painting. It may result in a masterpiece, but the probability is not high.

An important difficulty in the missile design area is that the progress of science has been so rapid that people have been forced to specialize and, as a result, designers with the breadth of background sufficient to handle a complete missile are almost non-existent. This leads to design by committee with the final product clearly showing the lumpy structure representing individual enthusiasms. In addition to covering fields of specialization, I believe our educational program should institute a process which will select a group of people with a high degree of imagination and the ability to comprehend a large variety of different areas, and institute a training program which will give them the range of skills required to undertake broad designs, such as a complete missile system. These people should understand, for a missile, such things as chemistry and propulsion, explosives, fuzing, power supplies (both battery and generator), hydraulic, electric, and pneumatic controls, aerodynamics, gyrodynamics, electronics, transistors, magnetic amplifiers, psychology and physiology, astronautics, and most importantly, be the victims of an all-inclusive curiosity. The emphasis in this type of training should be an understanding of the basic principles which, at present, limit our progress in any given direction. The details of calculations after the basic formulations are understood can well be left to machines or those people more interested in the limited areas. A course in the basic vocabularies of the many technical dialects would be valuable to these people who must converse intelligently in many fields.

I am sure it is obvious that we have very few people who can qualify as master designers for a missile system. The time required for one many to comprehend all the problems involved in a complete system will severely limit the rate of progress which is needed and the size of the overall program which can be undertaken. Because of these limitations, I am sure that we will always be faced with the problem of wanting to get equipment more rapidly than is possible. Therefore, we will always be forced to establish well planned "crash" programs involving the maximum rate of progress consistent with the techniques and technologies currently available. I feel strongly, however, that each such program should be accompanied by a small, artistic, and, perhaps, underfunded program which will look for the elegant solution rather than the obvious solution. The funding of such a program should not exceed 5\% of the crash program in order to prevent the kind of rigidity which sets in as soon as a program is pushed at such a high rate that feedback between the groups working on various parts is hampered. I suspect that such a program will not only save money but many times will come out ahead of the more massive program.

The Naval Ordnance Test Station is one of the organizations in the country which feels itself well-equipped to handle this artistic type of engineering design. We have the people and equipment necessary to investigate basic research problems, do development and development testing, do experimental production and testing, and have direct contact with the using services. Since we are part of the Navy organization, we are able to treat military design specifications with considerably more flexibility than is possible in a commercial organization working on a weapon design contract. WE have a very diverse group of people, running all the way from basic research scientists to operating military personnel, living in a community of about 12,000 people. This type of community life, which promotes the rapid exchange of technical information, generates the kinds of radically new ideas which result from the cross-breeding of the ideas of people with basically different backgrounds. Because we are close to facilities for testing these ideas in either the laboratory or on the ranges and have the tools available to construct the hardware necessary for these tests, we have at hand the mechanism to rapidly sort out the most useful ideas.

I have with me a movie entitled "Expanding Frontiers in Ordnance" which slows some of our facilities and illustrates the variety of skills and techniques which we have available for incorporation into the design of new military devices. At the conclusion of the film, I will be happy to answer any questions you might have.

\end{document}
