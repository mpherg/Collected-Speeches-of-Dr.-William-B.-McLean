
\documentclass{memoir}
\usepackage[utf8]{inputenc}
\usepackage[english]{babel}
\usepackage{csquotes}

\title{%
  INTEGRATED FACILITIES FOR WEAPONS DEVELOPMENT \\
  \small Presented at the Joint IRE-RTCA Spring 1957 Meeting \\
         by \\
         Wm. B. McLean, Technical Director \\
         U.S. Naval Ordnance Test Station \\
         China Lake, California
}

\author{Wm. B. McLean}
\date{Spring 1957}

\begin{document}

\maketitle

In considering the things concerning SIDEWINDER about which I might talk to you tonight, I found myself in what seems to be our usual security position of not being able to release as much about the weapon system as you have already read in the various newspapers and magazines. It therefore seems to me to be desirable to concentrate the remarks tonight on the organization which I feel has made the development of SIDEWINDER possible, rather than on the characteristics of the weapon itself.

When I was working at the National Bureau of Standards in Washington, D. C. during World War II, I always felt in the position of not having sufficient grasp of the various factors entering into the weapon system to really make an effective contribution. We were limited on one hand by the military specifications established by the Department of Defense, and on the other hand by the difficulties of accomplishing test operations after the first experimental models were completed. The second problem was complicated considerably by the fact that we usually spent weeks at Eglin or Dahlgren waiting for the weather to clear before we could get off the simplest tests of an explosive type equipment.

Near the end of the war I heard about the establishment of the Naval Ordnance Test station located in the middle of the Mojave Desert where the sky was always clear. I understood the organization would be part of the Department of Defense and could influence the setting up of military specifications. I felt the opportunity for doing effective military development would be ideal under these conditions. During the past twelve years of operation at the Naval Ordnance Test Station, I have found the combination of the laboratory closely associated with testing facilities to be an ideal situation for the stimulation and carrying out of new techniques for ordnance equipment. I think you will all recognize that at the present time the military organization has a very difficult responsibility with regards to seeing that the country is properly equipped with the best possible weapons to fight a war or to prevent a war. This problem requires an intimate knowledge of such a large number of technical possibilities that it is very difficult for any one person to achieve sufficient breadth of understanding.

Good design represents a process involving a continuous series of compromises among the military requirements, research and developmental possibilities, good engineering practice, and operability in the field. I am sure you are all familiar with the fact that the development man always considers the research man impractical, and the engineering and production man wishes the development people would be more concerned with the problems of production, and, as the system proceeds into test, the testing personnel cannot understand why the equipment cannot be made more reliable, and, finally, when it gets into service use it is always too complicated. The Naval Ordnance Test Station is organized on such a basis that this type of complaint can readily feed back between the various groups concerned. By living and working together, the operational military man, the research, development, engineering, and test personnel are all mutually able to influence each other and, we believe, all generate broader points of view. As the understanding of each other's problem grows, we believe the design compromises will be more carefully evaluated and better overall specifications and designs will result.

I do not believe that in the field of ordnance very many organizations, such as the naval Ordnance Test Station, can be established due to the large requirements in both facilities and territory. It is therefore incumbent on us to make our influence as widespread as possible. We want to keep our workload confined to these problems which generally would represent a high risk to any industrial organization due to the lack of reasonable and specific specifications. It is in this area that the quick interplay of research, design, and test, and their influence on requirements can be most fruitful. We believe that the end process of our operations can be a reasonable and workable specification which can be turned over to industrial organizations for production at a much lower cost than might otherwise be involved.

WE are gradually becoming aware of the widespread effect which can be achieved through the training of people. The broadening of viewpoints produced in both military officers and civilian scientists will remain effective as they proceed to other assignments. The military man moves on to such jobs on a regular schedule. I am sure that the whole industry represented here tonight has experience the fact that, while not as regularly scheduled, the movement of scientific manpower in the civilian areas can be as rapid, particularly under the competitive conditions existing in the Los Angeles area. We feel that this movement can work considerable hardship in particular because through it we can contribute to better understanding of military problems in industrial organizations. In fact, we have considered that as people leave the Naval Ordnance Test Station we should perhaps provide them with a degree in ordnance science as recognition of their completion of a period of study of the many problems enterinto the design of military equipment.

I would now like to show you some pictures of our operations at the Naval Ordnance Test Station together with some firing sequences showing SIDEWINDER in fleet use. I have also picked out some pictures of flutter tests on the SNORT track which I believe will be of particular interest to this group. Following this, we will have the story of the development of the DART tow target which, while it represents a very simple development program, illustrates very well the kinds of interchange which occur in all of our projects between the testing operations and further research and design.

\end{document}
